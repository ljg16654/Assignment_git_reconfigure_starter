% Created 2021-04-11 日 11:24
% Intended LaTeX compiler: pdflatex
\documentclass[a4paper]{article}
\usepackage[utf8]{inputenc}
\usepackage[T1]{fontenc}
\usepackage{graphicx}
\usepackage{grffile}
\usepackage{longtable}
\usepackage{wrapfig}
\usepackage{rotating}
\usepackage[normalem]{ulem}
\usepackage{amsmath}
\usepackage{textcomp}
\usepackage{amssymb}
\usepackage{capt-of}
\usepackage{hyperref}
\usepackage{amsthm}
\usepackage[most]{tcolorbox}
\usepackage{tikz}
\usepackage{physics}
\usepackage[margin=2.5cm]{geometry}
\usepackage{mathtools}
\newcounter{theorem_cnt}[section]
\newcounter{definition_cnt}[section]
\newcounter{proposition_cnt}[section]
\newtcbtheorem{theorem}{Theorem}
{breakable, enhanced, before title={\stepcounter{theorem_cnt}}, colback=gray!5, colframe=green!35!black, fonttitle=\bfseries,
attach boxed title to top left={xshift=5mm, yshift*=-\tcboxedtitleheight/2},
boxed title style={colback=blue!35!green}, separator sign={\ $\blacktriangleright$}}
{theorem}
\newtcbtheorem[number within=theorem_cnt]{lemma}{Lemma}{breakable, enhanced, colback=gray!5, colframe=blue!35!black, fonttitle=\bfseries,
coltitle=white, separator sign={\ $\blacktriangleright$}}
{lemma}
\newtcbtheorem{proposition}{Proposition}
{breakable, enhanced, before title = {\stepcounter{proposition_cnt}}, colback=gray!5, colframe=blue!35!black, fonttitle=\bfseries,
coltitle=white, separator sign={\ $\blacktriangleright$}}
{proposition}
\newtcbtheorem{comdefinition}{Definition}
{breakable, enhanced, before title={\stepcounter{definition_cnt}}, colback=gray!5, colframe=green!35!black, fonttitle=\bfseries,
attach boxed title to top left={xshift=5mm, yshift*=-\tcboxedtitleheight/2},
boxed title style={colback=blue!35!green}, separator sign={\ $\blacktriangleright$}}
{definition}
\newtcbtheorem{component}{Component}
{breakable, enhanced, before title={\stepcounter{definition_cnt}}, colback=gray!5, colframe=green!35!black, fonttitle=\bfseries,
attach boxed title to top left={xshift=5mm, yshift*=-\tcboxedtitleheight/2},
boxed title style={colback=blue!35!green}, separator sign={\ $\blacktriangleright$}}
{component}
\newtcolorbox{fact}[1][Fact]
{breakable, fonttitle=\bfseries, colback=white, title=#1, titlebox=visible, colback=gray!5}
\newtcolorbox{notation}[1][Notation]
{breakable, fonttitle=\bfseries, colback=white, title=#1, titlebox=visible, colback=gray!5}
\newtcolorbox{remark}[1][Remark]
{breakable, fonttitle=\bfseries, colback=white, title=#1, titlebox=visible, colback=gray!5}
\newtcbtheorem[number within=tcnt]{corollary}{Corollary}{breakable, enhanced, colback=gray!5, colframe=blue!35!black, fonttitle=\bfseries,
coltitle=white, separator sign={\ $\blacktriangleright$}}
{corollary}
\author{Jigang Li}
\date{\today}
\title{Assignment1\\\medskip
\large Basic usage of \texttt{git} and dynamic reconfigure}
\hypersetup{
 pdfauthor={Jigang Li},
 pdftitle={Assignment1},
 pdfkeywords={\author{Jigang Li}},
 pdfsubject={},
 pdfcreator={Emacs 28.0.50 (Org mode 9.4.5)}, 
 pdflang={English}}
\begin{document}

\maketitle

\section{Git}
\label{sec:orgadf6835}

If you haven't been familiar with basic usage of \texttt{Git}, please issue the
following command in your terminal:

\begin{verbatim}
man git
\end{verbatim}

The manpage will guide you to \texttt{gittutorial(7)} and \texttt{giteveryday(7)}, which
are great material for getting started with \texttt{Git}.

If you haven't used \texttt{man} before, you can get started with \texttt{man man}!

\section{Dynamic reconfigure}
\label{sec:org348f938}

Please read  \href{http://wiki.ros.org/dynamic\_reconfigure/Tutorials}{ROS tutorial on dynamic reconfigure} and create a package
named \texttt{dynamic\_tutorials} with the following components:

\begin{component}{A dynamically reconfigureable node} {}
The reconfigureable parameters should include
\begin{itemize}
\item a \texttt{enumuration} specifying size of the polygon (distance from the
center to any vertex of it in pixel).
\item an integer specifying number of sides of the polygon.
\end{itemize}
\end{component}

\begin{component}{A server} {}
The server should
\begin{itemize}
\item print the received parameters
\item render a polygon with \texttt{OpenCV}. Its shape is specified in the
dynamically reconfigureable node you've written.
\end{itemize}
\end{component}

\section{Submission}
\label{sec:org6d5d073}

We encourage the usage of github classroom to submit your
assignment. The \textbf{package} you created for the assignment is the git
repository. If you meet any problems, please send emails to
hcimu@umich.edu.
\end{document}